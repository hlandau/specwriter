\xmlregisterdocumentsetup{schspec-iso}{xs:base}
\xmlregisterns{s}{https://www.devever.net/ns/schspec-iso}
%\xmlregisterns{mml}

\startxmlsetups xs:base
  \xmlsetsetup{#1}{s:top|s:doc|s:clause|s:annex|s:termdef|s:term|s:iref|s:hdr|s:ul|s:ol|s:li|s:procn|s:proword|s:p|s:fig|s:fig-output|s:figure|s:docmain|s:docannex|s:regblock|s:reg|s:field|s:bitrange}{xs:*}
  \xmlsetsetup{#1}{s:clause/s:clause}{xs:clause2}
  \xmlsetsetup{#1}{s:clause/s:clause/s:clause}{xs:clause3}
  \xmlsetsetup{#1}{s:clause/s:clause/s:clause/s:clause}{xs:clause4}
  \xmlsetsetup{#1}{s:clause/s:clause/s:clause/s:clause/s:clause}{xs:clause5}

  \xmlsetsetup{#1}{s:annex/s:clause}{xs:clause2}
  \xmlsetsetup{#1}{s:annex/s:clause/s:clause}{xs:clause3}
  \xmlsetsetup{#1}{s:annex/s:clause/s:clause/s:clause}{xs:clause4}
  \xmlsetsetup{#1}{s:annex/s:clause/s:clause/s:clause/s:clause}{xs:clause5}
\stopxmlsetups

\startxmlsetups xs:top
  \xmlflush{#1}
\stopxmlsetups

\startxmlsetups xs:doc
  \xdef\DocTitle{\xmltext{#1}{/s:docctl/s:docinfo/s:doctitle}}
  \setvariables[document][title={\DocTitle}]

  % Enable hyperlinking of contents, etc.
  \setupinteraction[state=start,color=,contrastcolor=,style=normal,title={\DocTitle},author={Hugo Landau}]
  \setupinteractionscreen[option=bookmark]
  \placebookmarks[chapter,section,subsection,subsubsection]


  %% Cover Page                                                             {{{1
  %%%%%%%%%%%%%%%%%%%%%%%%%%%%%%%%%%%%%%%%%%%%%%%%%%%%%%%%%%%%%%%%%%%%%%%%%%%%%%
  \begingroup
    %% context-xml changes the catcodes so that $ doesn't do anything for XML
    %% processing-related reasons. TikZ complains about this, so we need to
    %% temporarily restore it.
    \setcatcodetable\ctxcatcodes

    %% The cover page uses a background layer; the page itself is an empty page.
    \definelayer[cover][x=0mm,y=0mm,width=\paperwidth,height=\paperheight]
    \setlayer[cover]{%
    \starttikzpicture[overlay]
      %% Fill white page with white. Ensures TikZ doesn't relativise our coordinates.
      \fill[fill=white] (0,0) rectangle (210mm,-297mm);

      %% Document title.
      \draw[anchor=west] (5\bodyfontsize,-100mm) node {\bfd\DocTitle};

      %% The MAIN STRIKE insignia.
      \startscope[xshift=150.1mm,yshift=-281mm]
       \startscope[scale=0.060]
        %\draw (0,0) rectangle (1000mm,1000mm);
        \fill (863.200mm,596.700mm) -- (0,235.900mm) -- (0,0);
        \fill (863.200mm,596.700mm) -- (520.600mm,783.400mm) -- (522.600mm,750.400mm);
        %\fill[fill=black] (0,0) rectangle (1000mm,1000mm);
        %\fill[fill=white] (863.200mm,596.700mm) -- (0,235.900mm) -- (0,0);
        %\fill[fill=white] (863.200mm,596.700mm) -- (520.600mm,783.400mm) -- (522.600mm,750.400mm); 
       \stopscope
      \stopscope

      %% Document control information.
      \draw[anchor=south west,align=left] (5\bodyfontsize,-281mm) node[fill=white,draw=black,inner sep=5mm] {{\tt //devever.net/...}\\ Revision 2020-01-01 ({\tt deadbeef}) };
    \stoptikzpicture%
    }
    \setupbackgrounds[page][background=cover]

    %% The empty page.
    \startstandardmakeup[pagestate=start] % Ensure page is numbered
    \stopstandardmakeup
  \endgroup


  %% Copyright Page                                                         {{{1
  %%%%%%%%%%%%%%%%%%%%%%%%%%%%%%%%%%%%%%%%%%%%%%%%%%%%%%%%%%%%%%%%%%%%%%%%%%%%%%
  \begingroup
    \startstandardmakeup[pagestate=start] % Ensure page is numbered
    \vfill

    Feedback, errata: {\tt hlandau@devever.net}

    \vskip10mm

    {\tt //devever.net/...}\endgraf

    Revision 2020-01-01 ({\tt deadbeef})
    \vskip5mm

    System \ConTeXt{}
    \vskip5mm

    © 2020 Hugo Landau

    \stopstandardmakeup
  \endgroup

  \title{Table of Contents}
  \placecontent


  %%                                                                        {{{1
  %%%%%%%%%%%%%%%%%%%%%%%%%%%%%%%%%%%%%%%%%%%%%%%%%%%%%%%%%%%%%%%%%%%%%%%%%%%%%%

  %\title{\DocTitle}
  %\title{\xmlfirst{#1}{/s:docctl/s:docinfo/s:doctitle}}
  \xmlfirst{#1}{/s:docbody}
\stopxmlsetups

\startxmlsetups xs:termdef
  \masksection{\xmlfirst{#1}{/s:hdr/s:title}}
  \reference[term:\xmlattribute{#1}{.}{sym}]{}
  \reference[term:\xmlattribute{#1}{.}{sym}s]{}
  \xmlflush{#1}
\stopxmlsetups

\startxmlsetups xs:clause
  \chapter[clause:\xmlattribute{#1}{.}{id}]{\xmlfirst{#1}{/s:hdr/s:title}}
  %\reference[clause:\xmlattribute{#1}{.}{id}]{DaDa}
  \xmlflush{#1}
\stopxmlsetups

\startxmlsetups xs:clause2
  \section[clause:\xmlattribute{#1}{.}{id}]{\xmlfirst{#1}{/s:hdr/s:title}}
  %\reference[clause:\xmlattribute{#1}{.}{id}]{DaDa}
  \xmlflush{#1}
\stopxmlsetups

\startxmlsetups xs:clause3
  \subsection[clause:\xmlattribute{#1}{.}{id}]{\xmlfirst{#1}{/s:hdr/s:title}}
  %\reference[clause:\xmlattribute{#1}{.}{id}]{DaDa}
  \xmlflush{#1}
\stopxmlsetups

\startxmlsetups xs:clause4
  \subsubsection[clause:\xmlattribute{#1}{.}{id}]{\xmlfirst{#1}{/s:hdr/s:title}}
  %\reference[clause:\xmlattribute{#1}{.}{id}]{DaDa}
  \xmlflush{#1}
\stopxmlsetups

\startxmlsetups xs:clause5
  \subsubsubsection[clause:\xmlattribute{#1}{.}{id}]{\xmlfirst{#1}{/s:hdr/s:title}}
  %\reference[clause:\xmlattribute{#1}{.}{id}]{DaDa}
  \xmlflush{#1}
\stopxmlsetups

\startxmlsetups xs:annex
  \chapter[clause:\xmlattribute{#1}{.}{id}]{\xmlfirst{#1}{/s:hdr/s:title}}
  %\reference[clause:\xmlattribute{#1}{.}{id}]{DaDa}
  \xmlflush{#1}
\stopxmlsetups

\startxmlsetups xs:docmain
  \startbodymatter
  \xmlflush{#1}
  \stopbodymatter
\stopxmlsetups

\startxmlsetups xs:docannex
  \startappendices
  \xmlflush{#1}
  \stopappendices
\stopxmlsetups


\startxmlsetups xs:hdr
\stopxmlsetups

\startxmlsetups xs:ul
  \startitemize
    \xmlflush{#1}
  \stopitemize
\stopxmlsetups

\startxmlsetups xs:ol
  \startitemize[n]
    \xmlflush{#1}
  \stopitemize
\stopxmlsetups

\startxmlsetups xs:li
  \item \xmlflush{#1}
\stopxmlsetups

\startxmlsetups xs:procn
  {\sc \xmlflush{#1}}
\stopxmlsetups

\startxmlsetups xs:proword
  {\bf \xmlflush{#1}}
\stopxmlsetups

\startxmlsetups xs:term
  \goto{\color[termgreen]{\xmlflush{#1}}}[term:\xmlattribute{#1}{.}{sym}]
\stopxmlsetups

\startxmlsetups xs:iref
  \goto{§ \ref[number][clause:\xmlattribute{#1}{.}{iref}] (\ref[title][clause:\xmlattribute{#1}{.}{iref}])}[clause:\xmlattribute{#1}{.}{iref}]
  %\in{§}{ (\about[clause:\xmlattribute{#1}{.}{iref}])}[clause:\xmlattribute{#1}{.}{iref}] 
  %\goto{\about[clause:\xmlattribute{#1}{.}{iref}]}[clause:\xmlattribute{#1}{.}{iref}]
\stopxmlsetups

\startxmlsetups xs:p
  \xmlflush{#1}\endgraf
\stopxmlsetups

\startxmlsetups xs:figure
  \placefigure{\xmlfirst{#1}{/s:hdr/s:title}}{ \xmlflush{#1} }
\stopxmlsetups

\startxmlsetups xs:fig
  \xmlfirst{#1}{/s:fig-outputs/s:fig-output[@type="application/pdf"]}
\stopxmlsetups

\startxmlsetups xs:fig-output
  \xdef\FigFn{\xmlattribute{#1}{.}{href}}
  \externalfigure[\FigPath\FigFn]
\stopxmlsetups

\startxmlsetups xs:regblock
  {\bf \xmlfirst{#1}{/s:title} (\xmlfirst{#1}{/s:mnem})}\endgraf

  \xmlall{#1}{/s:reg}
%       <regblock>
%                <mnem>CTLR</mnem>
%                <title>Controller Registers</title>
%                <reg>
%                  <mnem>CAP</mnem>
%                  <title>Controller Capabilities</title>
%                  <rb-offset>   0h</rb-offset>
%                  <rb-width>64</rb-width>
%                  <rb-access>ro</rb-access>
%                  <field>
%                    <mnem>MQES</mnem>
%                    <bitrange>
%                      <br-lo>0</br-lo>
%                      <br-hi>15</br-hi>
%                    </bitrange>
%                    <rb-access>ro</rb-access>
%                    <rb-title>Maximum Queue Entries Supported</rb-title>
%                    <rb-desc>This field indicates the maximum individual queue size that the controller supports. This value applies to each of the I/O Submission and I/O Completion Queues that host software may create. This is a zeroes-based value. The minimum value is 1, indicating two entries.</rb-desc>
%                  </field>
%
\stopxmlsetups

\startxmlsetups xs:reg
  \hskip.05\textwidth{\bf \xmlfirst{#1}{/s:title} (\xmlfirst{#1}{/s:mnem})}\hfill \xmlfirst{#1}{/s:rb-offset}
  \starttabulate[|rw(.1\textwidth)|p|]
  \xmlall{#1}{/s:field}
  \stoptabulate
\stopxmlsetups

\startxmlsetups xs:field
  \NC \xmlfirst{#1}{/s:bitrange} \NC {\bf \xmlfirst{#1}{/s:mnem}} (\xmlfirst{#1}{/s:rb-title}) \NC\NR
  \NC \NC \xmlfirst{#1}{/s:rb-desc} \NC\NR
\stopxmlsetups

\startxmlsetups xs:bitrange
  \xmldoifelse{#1}{/s:br-lo}{\xmlfirst{#1}{/s:br-lo}:\xmlfirst{#1}{/s:br-hi}}{\xmlfirst{#1}{/s:br-single}}
\stopxmlsetups

% TeX remains in the stone ages as far as font fallback support is concerned.
% We have to manually list fallback ranges. This list of ranges is incomplete,
% so this is terribly broken.
\definefontfallback[cjk][name:ArialUnicodeMS][cjkunifiedideographs,cjksymbolsandpunctuation,cjkcompatibility,cjkunifiedideographsextensiona,hiragana,katakana,halfwidthandfullwidthforms,cyrillic]

% Use Helvetica (actually, TeX Gyre Heros).
\definefontsynonym[Sans][heros][fallbacks=cjk]
\setupbodyfont[heros,10pt]

% A4.
\setuppapersize[A4]

% Layout.
\setuplayout[
  width=middle,
  height=middle,
  footer=3\bodyfontsize,
  header=2\bodyfontsize,
  headerdistance=\bodyfontsize,
  topspace=0.5\bodyfontsize,
  leftmargin=5\bodyfontsize,
  rightmargin=5\bodyfontsize
]

\setupstructure[state=start,method=auto]
\setupbackend[export=yes]
%\setupbackend[intent=ISO Uncoated,format=PDF/A-2a]
\setupreferencing[left={},right={}]

% Colours.
\setupcolors[state=start]
\definecolor[grey][r=.7,g=.7,b=.7]
\definecolor[termgreen][r=0,g=.3333,b=0]
\usemodule[tikz]


% Make chapter/section headings have the section number in the margin.
\unexpanded\def\PlaceSection#1#2%
  {\goodbreak
   \vbox
     {\localheadsetup
      \begstrut
      \inleftmargin{ #1}%
      #2}}

\setupheads[]
\setuphead[chapter][page=yes,before={\blank[10*big,force]},after=,style=\bfc,command=\PlaceSection]
\setuphead[section][style=\bf,before=\vskip3mm,after=,command=\PlaceSection]
\definehead[masksection][section]

% Set our header text.
\setupheadertexts[]
\setupheadertexts[\setups{text a}][][][\setups{text b}]

\startsetups[text a]
  \framed[width=\textwidth,frame=off,bottomframe=on]{\rlap{Page \pagenumber{} of \totalnumberofpages}
  \hfill
  {\sc \getmarking[chapter]}
  \hfill
  \llap{\sc \getvariable{document}{title}}}
\stopsetups
\startsetups[text b]
  \framed[width=\textwidth,frame=off,bottomframe=on]{\rlap{\sc \getvariable{document}{title}}
  \hfill
  {\sc \getmarking[section]}
  \hfill
  \llap{Page \pagenumber{} of \totalnumberofpages}}
\stopsetups

% Paragraph numbering.
\unexpanded\def\NumPara#1%
  {\inleftmargin{\color[grey]{¶#1}}%
  }

% Inter-paragraph spacing.
\setupwhitespace[small]

\starttext
\xmlprocessfile{schspec-iso}{\ArgInput}{}
\stoptext
